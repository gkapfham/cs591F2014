%!TEX root=cs591F2014-lab1.tex
% mainfile: cs591F2014-lab1.tex 
\input{labpre.tex}

\usepackage[compact]{titlesec}

\begin{document}
\MYTITLE{Laboratory Assignment One: Understanding and Customizing Android}
\MYHEADERS{Laboratory Assignment One}{Due: September 8 and 9, 2014}

\section*{Introduction}

Throughout this semester, we will use the Nexus 7 tablet and the Android operating system to learn more about mobile
computing and the development of mobile applications.  In this laboratory assignment, we will learn some of the basics of
the Android operating system and explore several applications that can customize and extend our interactions with the
Nexus 7 tablets.

\section*{Understanding Android}

In this class, we will use Android ``Jelly Bean'' 4.3.  To learn more about Android and the 4.3 version of this
operating system, you should visit and read the following Web sites: \url{http://www.android.com/} and
\url{http://www.android.com/whatsnew/} (students are encouraged to find and study other Android Web sites as well). Try
to find the user interface components and operating system features that these articles mention.  For instance, Android
has a notification area --- where is it?  How do you use it? Android also has a lock screen that supports the use of
widgets --- how do you add a widget to the lock screen?  Of course, there are many other important aspects of Android.
Make a document that gives a name and screenshot demonstrating each key aspect of Android.  What features of Android do you
find to be the most useful?  What features do you dislike?

Before we further explore and customize the Android environment, it is a good idea to test the tablets to ensure that
they are functioning correctly.  Using the Google Play store as the download source for these apps, please install
``AndroSensor'', ``GPS Test'', ``PixelTest'', and ``YAMTT''.  Using at least one screenshot for each app, please explain
their inputs, outputs, and behavior.  For instance, when you are using AndroSensor, try to change the environment of
the Nexus 7 tablet by covering the light sensor, rotating the device, and moving the cover of the magnetic case.  Do the
outputs produced by AndroSensor change when you interact with the tablet?

\section*{Customizing Android}

There are many apps that we can install to customize our interaction with the Android operating system and the Nexus 7
tablet.  For this laboratory assignment, you should install and configure both the ``DashClock Widget'' and ``Power
Toggles''.  Next, use DashClock to customize the lock screen on your Nexus 7 tablet, adding a DashClock widget and then
at least one extension to this widget.  Once you have finished installing and customizing DashClock, try to use 
Power Toggles to create a toggle panel in the notification area.  For example, you may want to add toggles for the
following aspects of the Android operating system: application data sync, wireless network connection, GPS signal,
bluetooth, volume, and the screen lock.  Again, make sure that you take a screen shot to demonstrate that you correctly
configured these apps.

To complete the assignment, you should turn in signed printouts of the: (i) description of Android's features, (ii)
results from the tablet tests, and (iii) outcomes from customizing Android.

\end{document}
