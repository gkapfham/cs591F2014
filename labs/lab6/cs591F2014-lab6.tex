%!TEX root=cs591F2014-lab6.tex
% mainfile: cs591F2014-lab6.tex 

\input{hwpre.tex}

\usepackage[compact]{titlesec}

\begin{document}
\MYTITLE{Laboratory Assignment Six: Exploring the Basics of AppInventor 2}
\MYHEADERS{Laboratory Assignment Six}{Due: November 10 and 11, 2014}

\vspace*{-.1in}
\section*{Introduction}

In this laboratory assignment, we will investigate the use of new tools for implementing mobile applications for the
Android operating system, with a focus on the AppInventor 2 system jointly developed by Google and the Massachusetts
Institute of Technology.  You can learn more about AppInventor 2 by visiting the Web site
\url{http://ai2.appinventor.mit.edu/} and browsing the chapters in the AppInventor 2 textbook available online at
\url{http://www.appinventor.org/book2}.

\section*{Configuring and Using AppInventor 2}

Before you visit the first Web site mentioned in the previous section, you should install a QR code reader, such as ``QR
Code Reader'' on your tablet. Now, go to this first Web site and log into it using your ``{\tt @allegheny.edu}''
account. What do you see on the screen? What seems to be some of the key features of AppInventor 2? Once you have
investigated this program's integrated development environment (IDE), please click on the link labelled ``Set up and
connect an Android device'' and study the details about making a WIFI connection. 

Following these instructions and ultimately using the QR code reader that you installed, scan the QR code created by
AppInventor 2. Please see the course instructor if you are not able to connect your tablet to the development
workstation through ``Allegheny Wireless''.




% In the next homework assignment, you will be asked to implement an app using AppInventor.  

As the final task for this homework assignment, you should browse the provided text book and several online sources as
you brainstorm examples of five apps that you could ultimately implement.

% \section*{Summary of the Required Deliverables}

% \vspace*{-.05in}
\noindent
To complete the assignment, you should turn in one copy of the following signed printouts: 
\vspace*{-.1in}

\begin{enumerate}
	\itemsep0em
	\item A description of the steps that you took to turn on the developer options
	\item Screen shots that convincingly demonstrate that you correctly configured AppInventor
	\item The complete description of five ideas for apps that you could create in AppInventor
\end{enumerate}

% \begin{enumerate}
%     \itemsep0em
% 
% 	\item Description of at least three automation tasks that your team could implement
% 		
% 	\item Complete documentation that describes the following aspects of your chosen automation tasks
% 
% \vspace*{-.05in}
% \begin{enumerate}
% 	\item Motivation for implementing the chosen tasks
% 	\item Inputs, outputs, and behavior of the tasks
% 	\item Implementation and testing choices for the tasks
% \end{enumerate}
% \vspace*{-.05in}
% 
% 	\item Report on the results from the user study that you performed with your automation tasks
% 
% 	\item Presentation slides for the in-class talk and demonstration that you will give in two weeks
% 
% \end{enumerate}
% 
\end{document}
