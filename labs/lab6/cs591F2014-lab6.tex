%!TEX root=cs591F2014-lab6.tex
% mainfile: cs591F2014-lab6.tex 

%!TEX root=cs591F2014-lab6.tex
% mainfile: cs591F2014-lab6.tex 

% CS 580 style
% Typical usage (all UPPERCASE items are optional):
%       \input 580pre
%       \begin{document}
%       \MYTITLE{Title of document, e.g., Lab 1\\Due ...}
%       \MYHEADERS{short title}{other running head, e.g., due date}
%       \PURPOSE{Description of purpose}
%       \SUMMARY{Very short overview of assignment}
%       \DETAILS{Detailed description}
%         \SUBHEAD{if needed} ...
%         \SUBHEAD{if needed} ...
%          ...
%       \HANDIN{What to hand in and how}
%       \begin{checklist}
%       \item ...
%       \end{checklist}
% There is no need to include a "\documentstyle."
% However, there should be an "\end{document}."
%
%===========================================================
\documentclass[11pt,twoside,titlepage]{article}
%%NEED TO ADD epsf!!
\usepackage{threeparttop}
\usepackage{graphicx}
\usepackage{latexsym}
\usepackage{color}
\usepackage{listings}
\usepackage{fancyvrb}
%\usepackage{pgf,pgfarrows,pgfnodes,pgfautomata,pgfheaps,pgfshade}
\usepackage{tikz}
\usepackage[normalem]{ulem}
\tikzset{
    %Define standard arrow tip
%    >=stealth',
    %Define style for boxes
    oval/.style={
           rectangle,
           rounded corners,
           draw=black, very thick,
           text width=6.5em,
           minimum height=2em,
           text centered},
    % Define arrow style
    arr/.style={
           ->,
           thick,
           shorten <=2pt,
           shorten >=2pt,}
}
\usepackage[noend]{algorithmic}
\usepackage[noend]{algorithm}
\newcommand{\bfor}{{\bf for\ }}
\newcommand{\bthen}{{\bf then\ }}
\newcommand{\bwhile}{{\bf while\ }}
\newcommand{\btrue}{{\bf true\ }}
\newcommand{\bfalse}{{\bf false\ }}
\newcommand{\bto}{{\bf to\ }}
\newcommand{\bdo}{{\bf do\ }}
\newcommand{\bif}{{\bf if\ }}
\newcommand{\belse}{{\bf else\ }}
\newcommand{\band}{{\bf and\ }}
\newcommand{\breturn}{{\bf return\ }}
\newcommand{\mod}{{\rm mod}}
\renewcommand{\algorithmiccomment}[1]{$\rhd$ #1}
\newenvironment{checklist}{\par\noindent\hspace{-.25in}{\bf Checklist:}\renewcommand{\labelitemi}{$\Box$}%
\begin{itemize}}{\end{itemize}}
\pagestyle{threepartheadings}
\usepackage{url}
\usepackage{wrapfig}
% removing the standard hyperref to avoid the horrible boxes
%\usepackage{hyperref}
\usepackage[hidelinks]{hyperref}
% added in the dtklogos for the bibtex formatting
\usepackage{dtklogos}
%=========================
% One-inch margins everywhere
%=========================
\setlength{\topmargin}{0in}
\setlength{\textheight}{8.5in}
\setlength{\oddsidemargin}{0in}
\setlength{\evensidemargin}{0in}
\setlength{\textwidth}{6.5in}
%===============================
%===============================
% Macro for document title:
%===============================
\newcommand{\MYTITLE}[1]%
   {\begin{center}
     \begin{center}
     \bf
     CMPSC 591\\Principles of Mobile Applications\\
     Fall 2014
     \medskip
     \end{center}
     \bf
     #1
     \end{center}
}
%================================
% Macro for headings:
%================================
\newcommand{\MYHEADERS}[2]%
   {\lhead{#1}
    \rhead{#2}
    %\immediate\write16{}
    %\immediate\write16{DATE OF HANDOUT?}
    %\read16 to \dateofhandout
    \def \dateofhandout {October 27 and 28, 2014}
    \lfoot{\sc Handed out on \dateofhandout}
    %\immediate\write16{}
    %\immediate\write16{HANDOUT NUMBER?}
    %\read16 to\handoutnum
    \def \handoutnum {7}
    \rfoot{Handout \handoutnum}
   }

%================================
% Macro for bold italic:
%================================
\newcommand{\bit}[1]{{\textit{\textbf{#1}}}}

%=========================
% Non-zero paragraph skips.
%=========================
\setlength{\parskip}{1ex}

%=========================
% Create various environments:
%=========================
\newcommand{\PURPOSE}{\par\noindent\hspace{-.25in}{\bf Purpose:\ }}
\newcommand{\SUMMARY}{\par\noindent\hspace{-.25in}{\bf Summary:\ }}
\newcommand{\DETAILS}{\par\noindent\hspace{-.25in}{\bf Details:\ }}
\newcommand{\HANDIN}{\par\noindent\hspace{-.25in}{\bf Hand in:\ }}
\newcommand{\SUBHEAD}[1]{\bigskip\par\noindent\hspace{-.1in}{\sc #1}\\}
%\newenvironment{CHECKLIST}{\begin{itemize}}{\end{itemize}}


\usepackage[compact]{titlesec}

\begin{document}
\MYTITLE{Laboratory Assignment Six: Exploring the Basics of AppInventor 2}
\MYHEADERS{Laboratory Assignment Six}{Due: November 10 and 11, 2014}

\vspace*{-.1in}
\section*{Introduction}

In this laboratory assignment, we will investigate the use of new tools for implementing mobile applications for the
Android operating system, with a focus on the AppInventor 2 system jointly developed by Google and the Massachusetts
Institute of Technology.  You can learn more about AppInventor 2 by visiting the Web site
\url{http://ai2.appinventor.mit.edu/} and browsing the chapters in the AppInventor 2 textbook available online at
\url{http://www.appinventor.org/book2}.

\section*{Configuring and Using AppInventor 2}

Before you visit the first Web site mentioned in the previous section, you should install a QR code reader, such as ``QR
Code Reader'' on your tablet. Now, go to this first Web site and log into it using your ``{\tt @allegheny.edu}''
account. What do you see on the screen? What seems to be some of the key features of AppInventor 2? Once you have
investigated this program's integrated development environment (IDE), please click on the link labelled ``Set up and
connect an Android device'' and study the tutorial that explains how to correctly make a WIFI connection. 

Following these instructions and ultimately using the QR code reader that you installed, scan the QR code created by
AppInventor 2 and download the correct app. At this point, you can create a new project called, for
example, ``AI2 Button Experiment''. Now, click ``Connect'' and then select ``AI Companion'' and scan the new QR code.
Please see the course instructor if you are not able to connect your tablet
to the development workstation through ``Allegheny Wireless''. 

Using the ``Designer'' environment, find the ``Button'' widget and add one to the mobile device in the middle of your
screen.  Please make sure that the button appears on both the development machine and your tablet---if it does not
appear on the tablet, then you may need to reconnect through the wireless network. Now, start the ``Blocks'' environment
by clicking this button on the right-hand side of the IDE. You can use this environment to define the behavior
of the button.  Please add ``puzzle pieces''---that is, visual program instructions---that will control the button. At
minimum, you should cause the button to change to red when it is clicked and then turn blue when it is long clicked; of
course, you can add new widgets and behaviors to your app as well.

Finally, you should create the app so that it can run, stand-alone, on your tablet. Click the ``Build'' menu item at the
top of the screen and then follow all of the instructions to download the app to your tablet by again scanning a QR
code.  As the final task for this laboratory assignment, you should online sources, such as
\url{http://appinventor.mit.edu/explore/ai2/tutorials.html}, to assist you in brainstorming examples of five apps that
you could create with AppInventor 2. To complete the assignment, you should turn in one copy of the following signed
printouts: 

% \section*{Summary of the Required Deliverables}

% \vspace*{-.05in}
% \noindent

\vspace*{-.1in}

\begin{enumerate}
	\itemsep0em
	\item A description of the steps that you took  to create your first AppInventor 2 project.
	\item Screen shots that convincingly demonstrate that you correctly configured AppInventor 2.
	\item The complete description of five ideas for apps that you could create in AppInventor 2.
\end{enumerate}

% \begin{enumerate}
%     \itemsep0em
% 
% 	\item Description of at least three automation tasks that your team could implement
% 		
% 	\item Complete documentation that describes the following aspects of your chosen automation tasks
% 
% \vspace*{-.05in}
% \begin{enumerate}
% 	\item Motivation for implementing the chosen tasks
% 	\item Inputs, outputs, and behavior of the tasks
% 	\item Implementation and testing choices for the tasks
% \end{enumerate}
% \vspace*{-.05in}
% 
% 	\item Report on the results from the user study that you performed with your automation tasks
% 
% 	\item Presentation slides for the in-class talk and demonstration that you will give in two weeks
% 
% \end{enumerate}
% 
\end{document}
