%!TEX root=cs591F2014-lab6.tex
% mainfile: cs591F2014-lab6.tex 

%!TEX root=cs591F2014-lab6.tex
% mainfile: cs591F2014-lab6.tex 

% CS 580 style
% Typical usage (all UPPERCASE items are optional):
%       \input 580pre
%       \begin{document}
%       \MYTITLE{Title of document, e.g., Lab 1\\Due ...}
%       \MYHEADERS{short title}{other running head, e.g., due date}
%       \PURPOSE{Description of purpose}
%       \SUMMARY{Very short overview of assignment}
%       \DETAILS{Detailed description}
%         \SUBHEAD{if needed} ...
%         \SUBHEAD{if needed} ...
%          ...
%       \HANDIN{What to hand in and how}
%       \begin{checklist}
%       \item ...
%       \end{checklist}
% There is no need to include a "\documentstyle."
% However, there should be an "\end{document}."
%
%===========================================================
\documentclass[11pt,twoside,titlepage]{article}
%%NEED TO ADD epsf!!
\usepackage{threeparttop}
\usepackage{graphicx}
\usepackage{latexsym}
\usepackage{color}
\usepackage{listings}
\usepackage{fancyvrb}
%\usepackage{pgf,pgfarrows,pgfnodes,pgfautomata,pgfheaps,pgfshade}
\usepackage{tikz}
\usepackage[normalem]{ulem}
\tikzset{
    %Define standard arrow tip
%    >=stealth',
    %Define style for boxes
    oval/.style={
           rectangle,
           rounded corners,
           draw=black, very thick,
           text width=6.5em,
           minimum height=2em,
           text centered},
    % Define arrow style
    arr/.style={
           ->,
           thick,
           shorten <=2pt,
           shorten >=2pt,}
}
\usepackage[noend]{algorithmic}
\usepackage[noend]{algorithm}
\newcommand{\bfor}{{\bf for\ }}
\newcommand{\bthen}{{\bf then\ }}
\newcommand{\bwhile}{{\bf while\ }}
\newcommand{\btrue}{{\bf true\ }}
\newcommand{\bfalse}{{\bf false\ }}
\newcommand{\bto}{{\bf to\ }}
\newcommand{\bdo}{{\bf do\ }}
\newcommand{\bif}{{\bf if\ }}
\newcommand{\belse}{{\bf else\ }}
\newcommand{\band}{{\bf and\ }}
\newcommand{\breturn}{{\bf return\ }}
\newcommand{\mod}{{\rm mod}}
\renewcommand{\algorithmiccomment}[1]{$\rhd$ #1}
\newenvironment{checklist}{\par\noindent\hspace{-.25in}{\bf Checklist:}\renewcommand{\labelitemi}{$\Box$}%
\begin{itemize}}{\end{itemize}}
\pagestyle{threepartheadings}
\usepackage{url}
\usepackage{wrapfig}
% removing the standard hyperref to avoid the horrible boxes
%\usepackage{hyperref}
\usepackage[hidelinks]{hyperref}
% added in the dtklogos for the bibtex formatting
\usepackage{dtklogos}
%=========================
% One-inch margins everywhere
%=========================
\setlength{\topmargin}{0in}
\setlength{\textheight}{8.5in}
\setlength{\oddsidemargin}{0in}
\setlength{\evensidemargin}{0in}
\setlength{\textwidth}{6.5in}
%===============================
%===============================
% Macro for document title:
%===============================
\newcommand{\MYTITLE}[1]%
   {\begin{center}
     \begin{center}
     \bf
     CMPSC 591\\Principles of Mobile Applications\\
     Fall 2014
     \medskip
     \end{center}
     \bf
     #1
     \end{center}
}
%================================
% Macro for headings:
%================================
\newcommand{\MYHEADERS}[2]%
   {\lhead{#1}
    \rhead{#2}
    %\immediate\write16{}
    %\immediate\write16{DATE OF HANDOUT?}
    %\read16 to \dateofhandout
    \def \dateofhandout {October 27 and 28, 2014}
    \lfoot{\sc Handed out on \dateofhandout}
    %\immediate\write16{}
    %\immediate\write16{HANDOUT NUMBER?}
    %\read16 to\handoutnum
    \def \handoutnum {7}
    \rfoot{Handout \handoutnum}
   }

%================================
% Macro for bold italic:
%================================
\newcommand{\bit}[1]{{\textit{\textbf{#1}}}}

%=========================
% Non-zero paragraph skips.
%=========================
\setlength{\parskip}{1ex}

%=========================
% Create various environments:
%=========================
\newcommand{\PURPOSE}{\par\noindent\hspace{-.25in}{\bf Purpose:\ }}
\newcommand{\SUMMARY}{\par\noindent\hspace{-.25in}{\bf Summary:\ }}
\newcommand{\DETAILS}{\par\noindent\hspace{-.25in}{\bf Details:\ }}
\newcommand{\HANDIN}{\par\noindent\hspace{-.25in}{\bf Hand in:\ }}
\newcommand{\SUBHEAD}[1]{\bigskip\par\noindent\hspace{-.1in}{\sc #1}\\}
%\newenvironment{CHECKLIST}{\begin{itemize}}{\end{itemize}}


\usepackage[compact]{titlesec}

\begin{document}
\MYTITLE{Laboratory Assignment Six: Exploring the Basics of AppInventor 2}
\MYHEADERS{Laboratory Assignment Six}{Due: November 10 and 11, 2014}

\vspace*{-.1in}
\section*{Introduction}

In this homework assignment, we will investigate the use of new tools for implementing mobile applications for the
Android operating system, with a focus on the AppInventor system developed jointly by Google and the
Massachusetts Institute of Technology.  You can learn more about AppInventor by visiting the Web site
\url{http://appinventor.mit.edu/} and browsing the chapters in the AppInventor textbook provided by the course
instructor.

\section*{Installing and Configuring AppInventor}



After configuring your tablet and visiting the Web site \url{http://appinventor.mit.edu/}, you should click the
``Invent'' button and log into AppInventor using your {\tt allegheny.edu} email address.  What do you see on the screen?
What are the key features of AppInventor? How does it work?



% In the next homework assignment, you will be asked to implement an app using AppInventor.  

As the final task for this homework assignment, you should browse the provided text book and several online sources as
you brainstorm examples of five apps that you could ultimately implement.

% \section*{Summary of the Required Deliverables}

% \vspace*{-.05in}
\noindent
To complete the assignment, you should turn in one copy of the following signed printouts: 
\vspace*{-.1in}

\begin{enumerate}
	\itemsep0em
	\item A description of the steps that you took to turn on the developer options
	\item Screen shots that convincingly demonstrate that you correctly configured AppInventor
	\item The complete description of five ideas for apps that you could create in AppInventor
\end{enumerate}

% \begin{enumerate}
%     \itemsep0em
% 
% 	\item Description of at least three automation tasks that your team could implement
% 		
% 	\item Complete documentation that describes the following aspects of your chosen automation tasks
% 
% \vspace*{-.05in}
% \begin{enumerate}
% 	\item Motivation for implementing the chosen tasks
% 	\item Inputs, outputs, and behavior of the tasks
% 	\item Implementation and testing choices for the tasks
% \end{enumerate}
% \vspace*{-.05in}
% 
% 	\item Report on the results from the user study that you performed with your automation tasks
% 
% 	\item Presentation slides for the in-class talk and demonstration that you will give in two weeks
% 
% \end{enumerate}
% 
\end{document}
