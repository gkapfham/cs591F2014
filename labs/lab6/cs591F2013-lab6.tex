\input{hwpre.tex}

\usepackage[compact]{titlesec}

\begin{document}
\MYTITLE{Homework Assignment Five: Installing and Configuring AppInventor}
\MYHEADERS{Homework Assignment Five}{Due: October 24 and 25, 2013}

\vspace*{-.1in}
\section*{Introduction}

In this homework assignment, we will investigate the use of new tools for implementing mobile applications for the
Android operating system, with a focus on the AppInventor system developed jointly by Google and the
Massachusetts Institute of Technology.  You can learn more about AppInventor by visiting the Web site
\url{http://appinventor.mit.edu/} and browsing the chapters in the AppInventor textbook provided by the course
instructor.

\section*{Installing and Configuring AppInventor}

Before you start this assignment, please make sure that you are sitting at a laboratory computer that has a USB cable
available for your use.  Furthermore, you must ensure that when you checkout your Nexus 7 tablet you also take the USB
cable that is used to connect it to the power supply.  After logging into your workstation, carefully connect your 
tablet to the workstation's USB cable.

When you go to the settings section of the Android operating system, you will notice that the ``Developer options'' menu
item is not currently available to you.  Since we need this item and the features that it provides, you will need to investigate
how to enable it for the ``Jelly Bean'' version of Android.  What trick did you discover?  Once the discover how to
enable this menu, please turn on USB debugging and take a screen shot to demonstrate that you have completed this task.

After configuring your tablet and visiting the Web site \url{http://appinventor.mit.edu/}, you should click the
``Invent'' button and log into AppInventor using your {\tt allegheny.edu} email address.  What do you see on the screen?
What are the key features of AppInventor? How does it work?

Now, we have to make sure that your tablet is correctly recognized by the Ubuntu operating system on your workstation.
Open a terminal window and type the command {\tt adb devices}.  If this command does not return output indicating that
your tablet is properly registered with the workstation, then further investigate this issue.  Next, you should return
to your tablet and install the app called ``MIT AI Companion''. See the instructor if you have problems with
these steps.

Once your tablet is properly registered with the workstation and the app is installed on the tablet, you should click a
button in the AppInventor window that will run the ``Blocks Editor''.  Next, make sure that AppInventor can connect to
your tablet.  Once you think that this process worked correctly, you should take a screen shot on both your workstation
and tablet.

% In the next homework assignment, you will be asked to implement an app using AppInventor.  

As the final task for this
homework assignment, you should browse the provided text book and several online sources as you brainstorm examples of
five apps that you could ultimately implement.

% \section*{Summary of the Required Deliverables}

% \vspace*{-.05in}
\noindent
To complete the assignment, you should turn in one copy of the following signed printouts: 
\vspace*{-.1in}

\begin{enumerate}
	\itemsep0em
	\item A description of the steps that you took to turn on the developer options
	\item Screen shots that convincingly demonstrate that you correctly configured AppInventor
	\item The complete description of five ideas for apps that you could create in AppInventor
\end{enumerate}

% \begin{enumerate}
%     \itemsep0em
% 
% 	\item Description of at least three automation tasks that your team could implement
% 		
% 	\item Complete documentation that describes the following aspects of your chosen automation tasks
% 
% \vspace*{-.05in}
% \begin{enumerate}
% 	\item Motivation for implementing the chosen tasks
% 	\item Inputs, outputs, and behavior of the tasks
% 	\item Implementation and testing choices for the tasks
% \end{enumerate}
% \vspace*{-.05in}
% 
% 	\item Report on the results from the user study that you performed with your automation tasks
% 
% 	\item Presentation slides for the in-class talk and demonstration that you will give in two weeks
% 
% \end{enumerate}
% 
\end{document}
