%!TEX root=cs591F2014-lab2.tex
% mainfile: cs591F2014-lab2.tex 
%!TEX root=cs591F2014-lab6.tex
% mainfile: cs591F2014-lab6.tex 

% CS 580 style
% Typical usage (all UPPERCASE items are optional):
%       \input 580pre
%       \begin{document}
%       \MYTITLE{Title of document, e.g., Lab 1\\Due ...}
%       \MYHEADERS{short title}{other running head, e.g., due date}
%       \PURPOSE{Description of purpose}
%       \SUMMARY{Very short overview of assignment}
%       \DETAILS{Detailed description}
%         \SUBHEAD{if needed} ...
%         \SUBHEAD{if needed} ...
%          ...
%       \HANDIN{What to hand in and how}
%       \begin{checklist}
%       \item ...
%       \end{checklist}
% There is no need to include a "\documentstyle."
% However, there should be an "\end{document}."
%
%===========================================================
\documentclass[11pt,twoside,titlepage]{article}
%%NEED TO ADD epsf!!
\usepackage{threeparttop}
\usepackage{graphicx}
\usepackage{latexsym}
\usepackage{color}
\usepackage{listings}
\usepackage{fancyvrb}
%\usepackage{pgf,pgfarrows,pgfnodes,pgfautomata,pgfheaps,pgfshade}
\usepackage{tikz}
\usepackage[normalem]{ulem}
\tikzset{
    %Define standard arrow tip
%    >=stealth',
    %Define style for boxes
    oval/.style={
           rectangle,
           rounded corners,
           draw=black, very thick,
           text width=6.5em,
           minimum height=2em,
           text centered},
    % Define arrow style
    arr/.style={
           ->,
           thick,
           shorten <=2pt,
           shorten >=2pt,}
}
\usepackage[noend]{algorithmic}
\usepackage[noend]{algorithm}
\newcommand{\bfor}{{\bf for\ }}
\newcommand{\bthen}{{\bf then\ }}
\newcommand{\bwhile}{{\bf while\ }}
\newcommand{\btrue}{{\bf true\ }}
\newcommand{\bfalse}{{\bf false\ }}
\newcommand{\bto}{{\bf to\ }}
\newcommand{\bdo}{{\bf do\ }}
\newcommand{\bif}{{\bf if\ }}
\newcommand{\belse}{{\bf else\ }}
\newcommand{\band}{{\bf and\ }}
\newcommand{\breturn}{{\bf return\ }}
\newcommand{\mod}{{\rm mod}}
\renewcommand{\algorithmiccomment}[1]{$\rhd$ #1}
\newenvironment{checklist}{\par\noindent\hspace{-.25in}{\bf Checklist:}\renewcommand{\labelitemi}{$\Box$}%
\begin{itemize}}{\end{itemize}}
\pagestyle{threepartheadings}
\usepackage{url}
\usepackage{wrapfig}
% removing the standard hyperref to avoid the horrible boxes
%\usepackage{hyperref}
\usepackage[hidelinks]{hyperref}
% added in the dtklogos for the bibtex formatting
\usepackage{dtklogos}
%=========================
% One-inch margins everywhere
%=========================
\setlength{\topmargin}{0in}
\setlength{\textheight}{8.5in}
\setlength{\oddsidemargin}{0in}
\setlength{\evensidemargin}{0in}
\setlength{\textwidth}{6.5in}
%===============================
%===============================
% Macro for document title:
%===============================
\newcommand{\MYTITLE}[1]%
   {\begin{center}
     \begin{center}
     \bf
     CMPSC 591\\Principles of Mobile Applications\\
     Fall 2014
     \medskip
     \end{center}
     \bf
     #1
     \end{center}
}
%================================
% Macro for headings:
%================================
\newcommand{\MYHEADERS}[2]%
   {\lhead{#1}
    \rhead{#2}
    %\immediate\write16{}
    %\immediate\write16{DATE OF HANDOUT?}
    %\read16 to \dateofhandout
    \def \dateofhandout {October 27 and 28, 2014}
    \lfoot{\sc Handed out on \dateofhandout}
    %\immediate\write16{}
    %\immediate\write16{HANDOUT NUMBER?}
    %\read16 to\handoutnum
    \def \handoutnum {7}
    \rfoot{Handout \handoutnum}
   }

%================================
% Macro for bold italic:
%================================
\newcommand{\bit}[1]{{\textit{\textbf{#1}}}}

%=========================
% Non-zero paragraph skips.
%=========================
\setlength{\parskip}{1ex}

%=========================
% Create various environments:
%=========================
\newcommand{\PURPOSE}{\par\noindent\hspace{-.25in}{\bf Purpose:\ }}
\newcommand{\SUMMARY}{\par\noindent\hspace{-.25in}{\bf Summary:\ }}
\newcommand{\DETAILS}{\par\noindent\hspace{-.25in}{\bf Details:\ }}
\newcommand{\HANDIN}{\par\noindent\hspace{-.25in}{\bf Hand in:\ }}
\newcommand{\SUBHEAD}[1]{\bigskip\par\noindent\hspace{-.1in}{\sc #1}\\}
%\newenvironment{CHECKLIST}{\begin{itemize}}{\end{itemize}}


\usepackage[compact]{titlesec}

\begin{document}
\MYTITLE{Laboratory Assignment Two: Using Conditional Logic in Automation Tasks}
\MYHEADERS{Laboratory Assignment Two}{Due: September 15 and 16, 2014}

\section*{Introduction}

Tasker is described as a ``total automation system for Android'' and is often one of the reasons why people purchase
Android-based tablets and mobile phones.  You can learn more about Tasker by visiting
\url{http://tasker.dinglisch.net/}. As you will see, the creator of Tasker described it as ``an application for Android
which performs Tasks (sets of Actions) based on Contexts (application, time, date, location, event, gesture) in
user-defined Profiles, or in clickable or timer [sic] home screen widgets.''  What are some ways in which you could use
Tasker to automate the use of your tablet?

\section*{Creating Notifications in Tasker}

Take some time to explore the ``Profiles'', ``Tasks'', ``Scenes'', and ``Vars'' tabs at the top of the screen in Tasker.
Using Tasker's in-application help system and any online resources that you find, write one or two sentences about the
meaning and purpose of each tab. Now, go to the ``Tasks'' tab and use the plus sign in the bottom-right corner to create
a new task called ``Display Notification''.  Next, find the Notify action and customize it so that it will display a
message.  Once you have completed this step, you can click the arrow in the bottom-left corner to run your task. You can
further tailor your notification task by changing the displayed message and/or the icon. 

\section*{Using Conditional Logic in Tasker}

Instead of having to load Tasker and click the run button every time you wanted to run your task, it would be better to
ensure that it automatically runs when a specified condition is true. To this end, please switch to the ``Profiles'' tab
and click the plus sign to create a new profile.  Next, pick a category from the pop-up menu that will describe the
context for your new profile.  For instance, you could specify that it is a certain time of the day or that the display
unlocked event just took place. Finally, you should add the ``Display Notification'' task to this profile.  Do you now
see a green arrow indicating that ``Display Notification'' will run when your specified context is true?  Please see the
instructor if you are not able to complete these steps.

Now that you have finished programming your first automation task, you should test it to ensure that it works correctly.
For instance, if you indicated that ``Display Notification'' should run when the screen is unlocked, then lock and
unlock the screen and see if the notification appears.  Once you have finished experimenting with your first task, you
should implement at least two others, for a total of three automation tasks. Each of the tasks that you implement should
display a notification when the specified condition(s) are true.  Students are encouraged to develop profile contexts that
contain multiple conditions that are joined by a plus sign representing the logical and operator. 

% \section*{Summary of the Required Deliverables}

To complete the assignment, you should turn in signed printouts of the: (i) list of at least six automation tasks
that you could implement in Tasker, (ii) description of the four tabs in Tasker's main screen, and (iii) screen shots
and descriptions of the behavior of three automation tasks.

\end{document}
