%!TEX root=cs591F2014-lab4.tex
% mainfile: cs591F2014-lab4.tex 

%!TEX root=cs591F2014-lab6.tex
% mainfile: cs591F2014-lab6.tex 

% CS 580 style
% Typical usage (all UPPERCASE items are optional):
%       \input 580pre
%       \begin{document}
%       \MYTITLE{Title of document, e.g., Lab 1\\Due ...}
%       \MYHEADERS{short title}{other running head, e.g., due date}
%       \PURPOSE{Description of purpose}
%       \SUMMARY{Very short overview of assignment}
%       \DETAILS{Detailed description}
%         \SUBHEAD{if needed} ...
%         \SUBHEAD{if needed} ...
%          ...
%       \HANDIN{What to hand in and how}
%       \begin{checklist}
%       \item ...
%       \end{checklist}
% There is no need to include a "\documentstyle."
% However, there should be an "\end{document}."
%
%===========================================================
\documentclass[11pt,twoside,titlepage]{article}
%%NEED TO ADD epsf!!
\usepackage{threeparttop}
\usepackage{graphicx}
\usepackage{latexsym}
\usepackage{color}
\usepackage{listings}
\usepackage{fancyvrb}
%\usepackage{pgf,pgfarrows,pgfnodes,pgfautomata,pgfheaps,pgfshade}
\usepackage{tikz}
\usepackage[normalem]{ulem}
\tikzset{
    %Define standard arrow tip
%    >=stealth',
    %Define style for boxes
    oval/.style={
           rectangle,
           rounded corners,
           draw=black, very thick,
           text width=6.5em,
           minimum height=2em,
           text centered},
    % Define arrow style
    arr/.style={
           ->,
           thick,
           shorten <=2pt,
           shorten >=2pt,}
}
\usepackage[noend]{algorithmic}
\usepackage[noend]{algorithm}
\newcommand{\bfor}{{\bf for\ }}
\newcommand{\bthen}{{\bf then\ }}
\newcommand{\bwhile}{{\bf while\ }}
\newcommand{\btrue}{{\bf true\ }}
\newcommand{\bfalse}{{\bf false\ }}
\newcommand{\bto}{{\bf to\ }}
\newcommand{\bdo}{{\bf do\ }}
\newcommand{\bif}{{\bf if\ }}
\newcommand{\belse}{{\bf else\ }}
\newcommand{\band}{{\bf and\ }}
\newcommand{\breturn}{{\bf return\ }}
\newcommand{\mod}{{\rm mod}}
\renewcommand{\algorithmiccomment}[1]{$\rhd$ #1}
\newenvironment{checklist}{\par\noindent\hspace{-.25in}{\bf Checklist:}\renewcommand{\labelitemi}{$\Box$}%
\begin{itemize}}{\end{itemize}}
\pagestyle{threepartheadings}
\usepackage{url}
\usepackage{wrapfig}
% removing the standard hyperref to avoid the horrible boxes
%\usepackage{hyperref}
\usepackage[hidelinks]{hyperref}
% added in the dtklogos for the bibtex formatting
\usepackage{dtklogos}
%=========================
% One-inch margins everywhere
%=========================
\setlength{\topmargin}{0in}
\setlength{\textheight}{8.5in}
\setlength{\oddsidemargin}{0in}
\setlength{\evensidemargin}{0in}
\setlength{\textwidth}{6.5in}
%===============================
%===============================
% Macro for document title:
%===============================
\newcommand{\MYTITLE}[1]%
   {\begin{center}
     \begin{center}
     \bf
     CMPSC 591\\Principles of Mobile Applications\\
     Fall 2014
     \medskip
     \end{center}
     \bf
     #1
     \end{center}
}
%================================
% Macro for headings:
%================================
\newcommand{\MYHEADERS}[2]%
   {\lhead{#1}
    \rhead{#2}
    %\immediate\write16{}
    %\immediate\write16{DATE OF HANDOUT?}
    %\read16 to \dateofhandout
    \def \dateofhandout {October 27 and 28, 2014}
    \lfoot{\sc Handed out on \dateofhandout}
    %\immediate\write16{}
    %\immediate\write16{HANDOUT NUMBER?}
    %\read16 to\handoutnum
    \def \handoutnum {7}
    \rfoot{Handout \handoutnum}
   }

%================================
% Macro for bold italic:
%================================
\newcommand{\bit}[1]{{\textit{\textbf{#1}}}}

%=========================
% Non-zero paragraph skips.
%=========================
\setlength{\parskip}{1ex}

%=========================
% Create various environments:
%=========================
\newcommand{\PURPOSE}{\par\noindent\hspace{-.25in}{\bf Purpose:\ }}
\newcommand{\SUMMARY}{\par\noindent\hspace{-.25in}{\bf Summary:\ }}
\newcommand{\DETAILS}{\par\noindent\hspace{-.25in}{\bf Details:\ }}
\newcommand{\HANDIN}{\par\noindent\hspace{-.25in}{\bf Hand in:\ }}
\newcommand{\SUBHEAD}[1]{\bigskip\par\noindent\hspace{-.1in}{\sc #1}\\}
%\newenvironment{CHECKLIST}{\begin{itemize}}{\end{itemize}}


\usepackage[compact]{titlesec}

\begin{document}
\MYTITLE{Laboratory Assignment Four: Implementing and Evaluating Automation Tasks}
\MYHEADERS{Laboratory Assignment Four}{Due: October 6 and 7, 2014}

\section*{Introduction}

In the last three laboratory assignments you have learned more about the Android operating system and the Tasker
automation environment. Since we will keep using Tasker in this laboratory assignment, you may want to learn more about
it by visiting \url{http://tasker.dinglisch.net/}.  Now, for this laboratory assignment, you will work in teams of four
or five students to describe, design, implement, test, and evaluate one or more Tasker automation tasks.  Over the next
two weeks, you will be responsible for organizing a team and then working with your team members to complete task(s)
that usefully automate the interaction with the Android operating system.

\section*{Automation Tasks for Android}

After brainstorming ideas with your team members, you should identify your top three candidates for automation tasks.
Next, you should prepare a document that explains your motivation for implementing one or more of these tasks.  After
completing this first part of the document, you should clearly specify the input(s), output(s), and behavior of your
tasks. Now, fairly divide up the implementation and testing effort associated with finishing your tasks. 

Once you have completed the implementation of the automation tasks, you should invite between three and five Allegheny
College students who are not members of this class to participate in a user study to evaluate your system.  After an
individual agrees to participate in your study, invite them to Alden Hall between 8:00 am and 4:00 pm so that they can
try out your tasks on one of the Nexus 7 tablets. When your friends are using your tasks, you should observe their
interactions and take notes.  Once they are finished interacting with the tablet, you should ask them a series of
questions to learn more about what they liked and disliked about your automation tasks.  

Finally, you should prepare slides for a short presentation and demonstration that you will give in class on the date
that this assignment is due.  The presentation should give an overview of the content that is in your written document,
briefly summarizing your motivation for implementing the tasks, explaining how the tasks were implemented, stating the inputs,
outputs, and behavior of the tasks, and highlighting the most important points from the feedback given by the users.

% \section*{Summary of the Required Deliverables}

% \vspace*{-.05in}
\noindent
To complete the assignment, you should turn in one copy of the following signed printouts: 
\vspace*{-.1in}
\begin{enumerate}
    \itemsep0em

	\item Description of at least three automation tasks that your team could implement
		
	\item Complete documentation that describes the following aspects of your chosen automation tasks

\vspace*{-.05in}
\begin{enumerate}
	\item Motivation for implementing the chosen tasks
	\item Inputs, outputs, and behavior of the tasks
	\item Implementation and testing choices for the tasks
\end{enumerate}
\vspace*{-.05in}

	\item Report on the results from the user study that you performed with your automation tasks

	\item Presentation slides for the in-class talk and demonstration that you will give in two weeks

\end{enumerate}

\end{document}
