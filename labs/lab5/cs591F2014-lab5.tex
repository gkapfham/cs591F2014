%!TEX root=cs591F2014-lab5.tex
% mainfile: cs591F2014-lab5.tex 

%!TEX root=cs591F2014-lab6.tex
% mainfile: cs591F2014-lab6.tex 

% CS 580 style
% Typical usage (all UPPERCASE items are optional):
%       \input 580pre
%       \begin{document}
%       \MYTITLE{Title of document, e.g., Lab 1\\Due ...}
%       \MYHEADERS{short title}{other running head, e.g., due date}
%       \PURPOSE{Description of purpose}
%       \SUMMARY{Very short overview of assignment}
%       \DETAILS{Detailed description}
%         \SUBHEAD{if needed} ...
%         \SUBHEAD{if needed} ...
%          ...
%       \HANDIN{What to hand in and how}
%       \begin{checklist}
%       \item ...
%       \end{checklist}
% There is no need to include a "\documentstyle."
% However, there should be an "\end{document}."
%
%===========================================================
\documentclass[11pt,twoside,titlepage]{article}
%%NEED TO ADD epsf!!
\usepackage{threeparttop}
\usepackage{graphicx}
\usepackage{latexsym}
\usepackage{color}
\usepackage{listings}
\usepackage{fancyvrb}
%\usepackage{pgf,pgfarrows,pgfnodes,pgfautomata,pgfheaps,pgfshade}
\usepackage{tikz}
\usepackage[normalem]{ulem}
\tikzset{
    %Define standard arrow tip
%    >=stealth',
    %Define style for boxes
    oval/.style={
           rectangle,
           rounded corners,
           draw=black, very thick,
           text width=6.5em,
           minimum height=2em,
           text centered},
    % Define arrow style
    arr/.style={
           ->,
           thick,
           shorten <=2pt,
           shorten >=2pt,}
}
\usepackage[noend]{algorithmic}
\usepackage[noend]{algorithm}
\newcommand{\bfor}{{\bf for\ }}
\newcommand{\bthen}{{\bf then\ }}
\newcommand{\bwhile}{{\bf while\ }}
\newcommand{\btrue}{{\bf true\ }}
\newcommand{\bfalse}{{\bf false\ }}
\newcommand{\bto}{{\bf to\ }}
\newcommand{\bdo}{{\bf do\ }}
\newcommand{\bif}{{\bf if\ }}
\newcommand{\belse}{{\bf else\ }}
\newcommand{\band}{{\bf and\ }}
\newcommand{\breturn}{{\bf return\ }}
\newcommand{\mod}{{\rm mod}}
\renewcommand{\algorithmiccomment}[1]{$\rhd$ #1}
\newenvironment{checklist}{\par\noindent\hspace{-.25in}{\bf Checklist:}\renewcommand{\labelitemi}{$\Box$}%
\begin{itemize}}{\end{itemize}}
\pagestyle{threepartheadings}
\usepackage{url}
\usepackage{wrapfig}
% removing the standard hyperref to avoid the horrible boxes
%\usepackage{hyperref}
\usepackage[hidelinks]{hyperref}
% added in the dtklogos for the bibtex formatting
\usepackage{dtklogos}
%=========================
% One-inch margins everywhere
%=========================
\setlength{\topmargin}{0in}
\setlength{\textheight}{8.5in}
\setlength{\oddsidemargin}{0in}
\setlength{\evensidemargin}{0in}
\setlength{\textwidth}{6.5in}
%===============================
%===============================
% Macro for document title:
%===============================
\newcommand{\MYTITLE}[1]%
   {\begin{center}
     \begin{center}
     \bf
     CMPSC 591\\Principles of Mobile Applications\\
     Fall 2014
     \medskip
     \end{center}
     \bf
     #1
     \end{center}
}
%================================
% Macro for headings:
%================================
\newcommand{\MYHEADERS}[2]%
   {\lhead{#1}
    \rhead{#2}
    %\immediate\write16{}
    %\immediate\write16{DATE OF HANDOUT?}
    %\read16 to \dateofhandout
    \def \dateofhandout {October 27 and 28, 2014}
    \lfoot{\sc Handed out on \dateofhandout}
    %\immediate\write16{}
    %\immediate\write16{HANDOUT NUMBER?}
    %\read16 to\handoutnum
    \def \handoutnum {7}
    \rfoot{Handout \handoutnum}
   }

%================================
% Macro for bold italic:
%================================
\newcommand{\bit}[1]{{\textit{\textbf{#1}}}}

%=========================
% Non-zero paragraph skips.
%=========================
\setlength{\parskip}{1ex}

%=========================
% Create various environments:
%=========================
\newcommand{\PURPOSE}{\par\noindent\hspace{-.25in}{\bf Purpose:\ }}
\newcommand{\SUMMARY}{\par\noindent\hspace{-.25in}{\bf Summary:\ }}
\newcommand{\DETAILS}{\par\noindent\hspace{-.25in}{\bf Details:\ }}
\newcommand{\HANDIN}{\par\noindent\hspace{-.25in}{\bf Hand in:\ }}
\newcommand{\SUBHEAD}[1]{\bigskip\par\noindent\hspace{-.1in}{\sc #1}\\}
%\newenvironment{CHECKLIST}{\begin{itemize}}{\end{itemize}}


\usepackage[compact]{titlesec}

\begin{document}
\MYTITLE{Laboratory Assignment Five: Implementing Graphical Apps with Generate}
\MYHEADERS{Laboratory Assignment Five}{Due: November 3 and 4, 2014}

\section*{Introduction}

In this laboratory assignment, you will explore the creation of interactive visualizations using the ``Generate:
Generative Art Tool'' app available for download from the Google Play store. This app allows you to write programs, in
the ``Context Free Art'' programming language, that produce colorful, interesting, and interactive visualizations. You can
learn more about this approach to programming by reviewing the in-app help system and the provided ``Context Free in a
Nutshell'' handout and, finally, by visiting the following Web site: \url{http://www.contextfreeart.org/}.

\section*{Exploring and Enhancing ``Generate'' Graphics}

To start this assignment, you must go to the Google Play store and download ``Generate: Generative Art Tool''.  Then,
you should review the aforementioned handout and Web site to learn more about how you can program in this language.  For
instance, you should notice that Generate will allow you to display circles and squares using the {\tt CIRCLE} and {\tt
SQUARE} directives.  You can also control the hue, saturation, and brightness of these shapes by respectively using the
``{\tt hue n}'', ``{\tt sat n}'', and ``{\tt b n}'' parameters. There is also a {\tt background} primitive that allows
you to specify the color to which the background will be assigned. The idea behind Generate programs is that there is a
start rule that is applied when the program is first run, specified by the {\tt startshape} command.  Additionally, you
can define {\tt TOUCH} rules that are activated only when the screen is touched.

One way to better understand how to program the Generate app is to review the ``Simple First'' example.  You can cause
this program to produce its visualization by clicking the ``Run'' button at the top of the screen. What is the output
that results from running this program? Can you understand how this program produces this output? Once you understand
this basic example, you should enhance it so that it sets the background to a different color than white. In addition,
you can change the rule that creates the square so that it now produces a circle. Next, you should also change the
source code that draws the circle so that it uses a different color than black. To learn more about the ``Context Free
Art'' programming language you should study the source code of and run at least five other provided programs; please see
the instructor if you have questions.

% \vspace*{-.05in}
\noindent
To complete the assignment, you should turn in one copy of the following signed printouts: 
% \vspace*{-.1in}

\begin{enumerate}
    \itemsep0em

	\item The output from running at least five of the Generate sample programs.
		
	\item A screenshot of the final source code of the modified ``Simple First'' example.

	\item The output from running the original and final versions of the ``Simple First'' program.

	\item A written explanation of how your final ``Simple First'' program actually works.

        \item A written description of the challenges that you faced when programming in Generate.

          \item A proposal for at least three Generate programs that you would like to implement.

\end{enumerate}

\end{document}
