\input{hwpre.tex}

\usepackage[compact]{titlesec}

\begin{document}
\MYTITLE{Laboratory Assignment Three: Integrating Tasker Automation Tasks}
\MYHEADERS{Laboratory Assignment Three}{Due: September 22 and 23, 2014}

\section*{Introduction}

The last laboratory assignment introduced Tasker, a ``total automation system for Android''.  Since we will continue to
use Tasker in this laboratory assignment, you may want to continue to learn more about it by visiting
\url{http://tasker.dinglisch.net/}. Moreover, in the first laboratory assignment we saw that it was possible to
customize the Android mobile operating system by installing programs such as DashClock and Power Toggles.  In this
laboratory assignment, we will learn more about how to invoke automation tasks from mobile apps such as DashClock.

\section*{Integrating Tasker and DashClock}

If you want to invoke Tasker automation tasks directly from DashClock, you will need to go to the Google Play Store and
install the app called ``DashClock Tasker Extension''.  Once you have installed this app, you can click DashClock's
configuration icon and start to add extensions that follow the naming convention ``Tasker \#1'' through ``Tasker \#3''.

% ; please note that the app is currently limited to four Tasker integrations.  

After you have placed a Tasker extension in the DashClock, you need to implement a Tasker automation task that can use
it properly.  Now, you should run the Tasker app and create a new task that uses the new DashClock Tasker plugin.  You
will need to click the ``Edit'' button for this task and configure it with an appropriate widget number, name, removal
preference, Tasker task, and icon. Upon saving this task, you need to create a profile that will trigger this task.

What type of Tasker automation task would you like to integrate into the DashClock?  As an example, you could add a new
DashClock feature that will display an icon and the label ``Turn off the Screen''.  When the user of the tablet clicks
this label after turning on the tablet, it will turn off the screen, thereby reducing the number of times that it would
be necessary to press the power button.  To ensure that this type of integration works correctly, you will need to add a
profile and a task to Tasker, following the previous instructions.  You can start this process by creating a profile
that will trigger a task when the display state is ``on''.  What must you do to finish this example?

\section*{Integrating Tasker and Power Toggles}

After running the Power Toggles app, you will notice that it has an icon called ``Notification Widget''.  As you will
remember from the first laboratory assignment, you can use this icon to customize the notification area of the Android
operating system.  Next, please click the ``Add Toggle'' button and, after clicking the ``Custom'' tab, pick the
``Tasker toggle''.  At this point, you should see a listing of all of the Tasker tasks that you have already programmed.
Take some time and implement a Tasker automation task that you can run from the notification area.

% \section*{Summary of the Required Deliverables}

To complete the assignment, you should turn in signed printouts of the: (i) list of at least four ways in which you
could integrate Tasker, DashClock, and Power Toggles, (ii) screen shots and descriptions that show one way to integrate
Tasker with each of DashClock and Power Toggles.

\end{document}
